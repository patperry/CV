% cv-us.tex
% $Id: cv-us.tex,v 1.28 2006/12/12 22:53:52 jrblevin Exp $
%
% LaTeX Curriculum Vitae Template
%
% Copyright (C) 2004-2006 Jason Blevins
%
% You may use use this document as a template to create your own CV
% and you may redistribute the source code freely. No attribution is
% required in any resulting documents, however, I do ask that you
% please leave this notice and the above URL in the source code if you
% choose to redistribute this file.
%
% Jason R. Blevins <jrblevin@sdf.lonestar.org>
% http://jrblevin.freeshell.org
% Durham, December 12, 2006
%
%%---------------------------------------------------------------------------%
%
% Notes:
%
% * Don't forget to change `pdfauthor' and `keywords' in the \hypersetup
%   section below.
%
% * To create a new page use: \newpage \opening
%
% * res.cls includes an \address{} command which can be used up to twice,
%   but my address is too long for the format it uses.
%
% * Alternate documentclass statement to put headings in margin:
%   \documentclass[margin,line,11pt,final]{res}
%
% * Can divide publication/presentation list into subsections by year:
%   \section{\bf\small\hspace{8mm}2006}
%
%%----------------------------------------------------------------------------%

\documentclass[overlapped,line,letterpaper]{res}

\usepackage{ifpdf}

\ifpdf
  \usepackage[pdftex]{hyperref}
\else
  \usepackage[hypertex]{hyperref}
\fi

\hypersetup{
  letterpaper,
  colorlinks,
  urlcolor=black,
  pdfpagemode=none,
  pdftitle={Curriculum Vitae},
  pdfauthor={Patrick O. Perry},
  pdfsubject={Curriculum Vitae},
  pdfkeywords={networks multivariate statistics}
}

%%===========================================================================%%

\begin{document}

%---------------------------------------------------------------------------
% Document Specific Customizations

% Make lists without bullets and with no indentation
\setlength{\leftmargini}{0em}
\renewcommand{\labelitemi}{}

% Use large bold font for printed name at top of pages
\renewcommand{\namefont}{\large\textbf}

%---------------------------------------------------------------------------

\name{Patrick O. Perry}

\begin{resume}

\begin{ncolumn}{2}
  Maxwell Dworkin Rm. 334       & Phone: (650) 619-5203 \\
  33 Oxford St                  & {\tt patperry@seas.harvard.edu} \\
  Cambridge, MA 02138           & {\tt \verb+http://people.seas.harvard.edu/~patperry/+}
\end{ncolumn}

%---------------------------------------------------------------------------

\section{\bf Education}

Ph.D. Statistics, Stanford University, 2009

M.S. Electrical Engineering, Stanford University, 2004

B.S. Mathematics, Stanford University, 2003, with honors

%%---------------------------------------------------------------------------%%

\section{\bf Fellowships and Awards}
\begin{itemize}
\item Burt and Deedee McMurtry Stanford Graduate Fellowship, 2004--2007
\end{itemize}

%---------------------------------------------------------------------------

\section{\bf Teaching Experience}

At Stanford University:

\begin{itemize}
\item Teaching Assistant, Spatial Statistics (Ph.D.), Spring 2009
\item Teaching Assistant, Statistical Consulting Workshop, Winter 2009
\item Teaching Assistant, Software for Data Analysis (Ph.D.), Winter 2008
\item Teaching Assistant, Monte Carlo Methods (Ph.D.), Fall 2007
\item Teaching Assistant, Modern Applied Statistics (Ph.D.), Fall 2006
\item Teaching Assistant, Statistical Consulting Workshop, Fall 2005
\item Teaching Assistant, Statistical Computing (M.S.), Spring 2005
\end{itemize}

%------------------------------------------------------------------------------

\section{\bf Research Experience}

\begin{format}
\title{l}\dates{r}\\
\employer{l}\location{r}\\
\body\\
\end{format}

\title{Postdoctoral Researcher}
\employer{Patrick\ J.\ Wolfe}
\location{Engineering and Applied Sciences, Harvard}
\dates{2009--present}
\begin{position}
  Modeling and inference for network data.
\end{position}

\title{Research Assistant}
\employer{Art\ B.\ Owen}
\location{Statistics, Stanford}
\dates{2006--2009}
\begin{position}
  Applied random matrix theory.  Latent variable methods.
\end{position}

\title{Research Assistant}
\employer{Gunnar\ E.\ Carlsson}
\location{Mathematics, Stanford}
\dates{2002--2004}
\begin{position}
  Computational topology software and data analysis.
\end{position}


\section{\bf Affiliations}


\begin{format}
\title{l}\dates{r}\\
\employer{l}\location{r}\\
%\body\\
\end{format}

\title{Visiting Scholar}
\employer{Institute for Quantitative Social Science}
\location{Harvard}
\dates{2010--present}
\begin{position}
\end{position}

\title{Member}
\employer{NSA Statistical Advisory Group}
\location{Stanford}
\dates{2010}
\begin{position}
\end{position}

\title{Member}
\employer{Geometrical/Spectral Analysis of Networks Working Group}
\location{SAMSI}
\dates{2010}
\begin{position}
\end{position}


%---------------------------------------------------------------------------

\section{\bf Employment}

\begin{format}
\title{l}\dates{r}\\
\employer{l}\location{r}\\
\body\\
\end{format}

\title{Consultant}
\employer{Insignia Environmental}
\location{Palo Alto, CA}
\dates{2009}
\begin{position}
    Bootstrap-based analysis of three experiments designed to estimate
    the environmental impact of installing new wind turbines in
    Contra Costa County, in California.
\end{position}

\title{Consultant}
\employer{Kingsford Capital Management}
\location{Pt. Richmond, CA}
\dates{2008--2009}
\begin{position}
    General statistics advice, help with data analysis and programming in R.
\end{position}

\title{Biostatistician}
\employer{Stanford School of Medicine}
\location{Stanford, CA}
\dates{2006--2007}
\begin{position}
    Data analysis and brief write-ups for studies performed by resident
    physicians under the supervision of Dr. Raj Mitra, Orthopaedic Surgery 
    Department.
\end{position}

%------------------------------------------------------------------------------

\section{\bf Publications}

``Point process modeling for directed interaction networks.''
With P.\ J.\ Wolfe.
\textit{Submitted to Journal of the Royal Statistical Society, Series B; arXiv:1011.1703v1 [stat.ME].}

``Minimax rank estimation for subspace tracking.''
With P.\ J.\ Wolfe.
\textit{IEEE Journal of Selected Topics in Signal Processing},
\textbf{4}(3):504--513, 2010.

``A rotation test to verify latent structure.''
With A.\ B.\ Owen.
\textit{Journal of Machine Learning Research},
\textbf{11}:603--624, 2010.

``Cross-validation for unsupervised learning.''
Supervised by A.\ B.\ Owen.
Ph.D. thesis, Stanford University. 2009.

``Bi-cross-validation of the SVD and the non-negative matrix factorization.''
With A.\ B.\ Owen.
\textit{Annals of Applied Statistics},
\textbf{3}(2):564--594, 2009.

``Efficacy of fluoroscopically guided steroid injections in the management of coccydynia.''
With R.\ Mitra and L.\ Cheung.
\textit{Pain Physician},
\textbf{10}(6):775--778, 2007.

``Application of simplicial sets to computational topology.''
Supervised by G.\ E.\ Carlsson.
Undergraduate honors thesis, Stanford University. 2003.

\section{\bf Working Papers}

``Dynamics of gang homicide in Chicago.''
With A.\ V. Papachristos, P.\ J.\ Wolfe.

``Implicitly Bayesian eigenvector estimation.''
With M.\ W.\ Mahoney.

``Latent variable network models: a unified view.''
With P.\ J.\ Wolfe


\section{\bf Presentations}

``Point process models for repeated interaction data.''
Harvard Statistics Colloquium, Cambridge, MA.  October 25, 2010.

``Point process models for pairwise interaction data.''
Stanford Statistics Seminar, Stanford, CA.  July 13, 2010.

``Point process models for pairwise interaction data.''
Conference on Statistical Modeling and Inference for Networks (Statworks), 
Bristol, UK.  June 30, 2010.

``A point process model for repeated interactions.''
MIT Lincoln Laboratory Seminar, Lexington, MA. June 11, 2010.

``Community and homophily in a corporate e-mail network.''
New England Statistics Symposium, Cambridge, MA.  April 17, 2010.

``Cross validation for principal components.''
BIRS Program on Sparse Random Structures: Analysis and Computation,
Banff, Canada.
January 28, 2010.

``Choosing how many principal components to keep.''
Stanford Statistics Seminar, Stanford, CA.  April 14, 2009.

``Cross validation for unsupervised learning.''
Wharton Statistics Seminar, Philadelphia, PA.  February 11, 2009.

``Cross validation for unsupervised learning.''
Hewlett Packard Labs Seminar, Palo Alto, CA.  September 26, 2008.


\section{\bf Posters}

``Point process models for pairwise interactions.''
With P.\ J.\ Wolfe.
SAMSI Program on Complex Networks Opening Workshop,
Research Triangle Park, NC,
August 29--September 1, 2010.

``Latent covariate detection and verification.''
With A.\ B.\ Owen.
Biomedical Computation at Stanford (BCATS) Symposium,
Stanford CA,
October 21, 2006.

``Latent covariate detection and verification: a case study in genetics.''
With A.\ B.\ Owen.
SAMSI Random Matrices Opening Workshop,
Research Triangle Park, NC,
September 17--20, 2006.


%------------------------------------------------------------------------------

%%===========================================================================%%
%\newpage
%\opening

\section{\bf Software}

{\texttt{bcv}}.
An \texttt{R} package for
cross-validation-based rank selection of SVD approximations. \\
\href{http://cran.r-project.org/web/packages/bcv/index.html}http://cran.r-project.org/web/packages/bcv/index.html.

{\texttt{hs-blas}}.
A linear algebra library for the Haskell programming language. \\
\href{http://hackage.haskell.org/cgi-bin/hackage-scripts/package/blas}http://hackage.haskell.org/cgi-bin/hackage-scripts/package/blas.

{\texttt{hs-monte-carlo}}.
A Monte Carlo simulation library for Haskell. \\
\href{http://hackage.haskell.org/cgi-bin/hackage-scripts/package/monte-carlo}http://hackage.haskell.org/cgi-bin/hackage-scripts/package/monte-carlo.

{\texttt{PLEX}}. With V.\ de Silva.
A computational topology library for \textsc{MATLAB} and C++.  \\
\href{http://comptop.stanford.edu/programs/plex.html}http://comptop.stanford.edu/programs/plex.html.

{\texttt{RMTstat}}. With I.\ Johnstone, Z.\ Ma, and M.\ Shahram.
An \texttt{R} package for distributions and statistics from random matrix theory. \\
%mostly related to large sample covariance matrices.
\href{http://cran.r-project.org/web/packages/RMTstat/index.html}http://cran.r-project.org/web/packages/RMTstat/index.html.


%------------------------------------------------------------------------------

\section{\bf Grant Proposals}

``Measuring, understanding, and responding to covert social networks.''
With P.\ J.\ Wolfe et al.
Office of Naval Research MURI Award 58153-MA-MUR.
2010.

%------------------------------------------------------------------------------

\section{\bf Peer-Review Experience}

Annals of Applied Statistics,
Annals of Statistics,
Chemometrics and Intelligent Laboratory Systems,
Journal of the Bernoulli Society,
Stochastic Processes and their Applications,
Transactions on Pattern Analysis and Machine Intelligence



%\section{\bf Presentations}
%
%``Solving Nonlinear Wave Equations with Mathematica,''
%North Carolina State University, April 23, 2003.

%---------------------------------------------------------------------------

%\section{\bf Service}
%
%\begin{itemize}
%\item Vice President, Society of Undergraduate Mathematics, North
%  Carolina State University, 2001--2003
%\item Vice President, Economics Graduate Student Council, Duke
%  University, 2006--2007
%\end{itemize}

%%---------------------------------------------------------------------------%%

%\begin{center}
%\vspace{\fill}\ \newline
%{\tiny \rm $ $RCSfile: cv-us.tex,v $ $ }
%{\tiny \rm $ $Date: 2006/12/12 22:53:52 $ $ }
%{\tiny \rm $ $Revision: 1.28 $ $ }
%\end{center}

\end{resume}

\end{document}

%%===========================================================================%%
