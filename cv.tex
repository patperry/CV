% cv-us.tex
% $Id: cv-us.tex,v 1.28 2006/12/12 22:53:52 jrblevin Exp $
%
% LaTeX Curriculum Vitae Template
%
% Copyright (C) 2004-2006 Jason Blevins
%
% You may use use this document as a template to create your own CV
% and you may redistribute the source code freely. No attribution is
% required in any resulting documents, however, I do ask that you
% please leave this notice and the above URL in the source code if you
% choose to redistribute this file.
%
% Jason R. Blevins <jrblevin@sdf.lonestar.org>
% http://jrblevin.freeshell.org
% Durham, December 12, 2006
%
%%---------------------------------------------------------------------------%
%
% Notes:
%
% * Don't forget to change `pdfauthor' and `keywords' in the \hypersetup
%   section below.
%
% * To create a new page use: \newpage \opening
%
% * res.cls includes an \address{} command which can be used up to twice,
%   but my address is too long for the format it uses.
%
% * Alternate documentclass statement to put headings in margin:
%   \documentclass[margin,line,11pt,final]{res}
%
% * Can divide publication/presentation list into subsections by year:
%   \section{\bf\small\hspace{8mm}2006}
%
%%----------------------------------------------------------------------------%

\documentclass[overlapped,line,letterpaper]{res}

\usepackage{ifpdf}

\ifpdf
  \usepackage[pdftex]{hyperref}
\else
  \usepackage[hypertex]{hyperref}
\fi

\hypersetup{
  letterpaper,
  colorlinks,
  urlcolor=black,
  pdfpagemode=none,
  pdftitle={Curriculum Vitae},
  pdfauthor={Patrick O. Perry},
  pdfcreator={$ $Id: cv-us.tex,v 1.28 2006/12/12 22:53:52 jrblevin Exp $ $},
  pdfsubject={Curriculum Vitae},
  pdfkeywords={statistics applied matrix decomposition cross validation}
}

%%===========================================================================%%

\begin{document}

%---------------------------------------------------------------------------
% Document Specific Customizations

% Make lists without bullets and with no indentation
\setlength{\leftmargini}{0em}
\renewcommand{\labelitemi}{}

% Use large bold font for printed name at top of pages
\renewcommand{\namefont}{\large\textbf}

%---------------------------------------------------------------------------

\name{Patrick O. Perry}

\begin{resume}

\begin{ncolumn}{2}
  Maxwell Dworkin Rm. 334       & Phone: (650) 619-5203 \\
  33 Oxford St                  & {\tt patperry@seas.harvard.edu} \\
  Cambridge, MA 02138           & {\tt \verb+http://people.seas.harvard.edu/~patperry/+}
\end{ncolumn}

%---------------------------------------------------------------------------

\section{\bf Education}

Ph.D. Statistics, Stanford University, 2009

M.S. Electrical Engineering, Stanford University, 2004

B.S. Mathematics, Stanford University, 2003, with honors

%%---------------------------------------------------------------------------%%

\section{\bf Fellowships and Awards}
\begin{itemize}
\item Burt and Deedee McMurtry Stanford Graduate Fellowship, 2004--2007
\end{itemize}

%---------------------------------------------------------------------------

\section{\bf Teaching Experience}

At Stanford University:

\begin{itemize}
\item Teaching Assistant, Spatial Statistics (Ph.D.), Spring 2009
\item Teaching Assistant, Statistical Consulting Workshop, Winter 2009
\item Teaching Assistant, Software for Data Analysis (Ph.D.), Winter 2008
\item Teaching Assistant, Monte Carlo Methods (Ph.D.), Fall 2007
\item Teaching Assistant, Modern Applied Statistics (Ph.D.), Fall 2006
\item Teaching Assistant, Statistical Consulting Workshop, Fall 2005
\item Teaching Assistant, Statistical Computing (M.S.), Spring 2005
\end{itemize}

%------------------------------------------------------------------------------

\section{\bf Research Experience}

\begin{format}
\title{l}\dates{r}\\
\employer{l}\location{r}\\
\body\\
\end{format}

\title{Postdoctoral Researcher}
\employer{Patrick\ J.\ Wolfe}
\location{Engineering and Applied Sciences, Harvard}
\dates{2009--present}
\begin{position}
  Modeling and inference for social networks.
\end{position}

\title{Research Assistant}
\employer{Art\ B.\ Owen}
\location{Statistics, Stanford}
\dates{2006--2009}
\begin{position}
  Cross validation for matrix decompositions.  Applied random matrix theory.
\end{position}

\title{Research Assistant}
\employer{Gunnar\ E.\ Carlsson}
\location{Mathematics, Stanford}
\dates{2002--2004}
\begin{position}
  Computational topology software and data analysis.
\end{position}


\section{\bf Affiliations}


\begin{format}
\title{l}\dates{r}\\
\employer{l}\location{r}\\
\body\\
\end{format}

\title{Fellow}
\employer{Institute for Quantitative Social Science}
\location{Harvard}
\dates{2010--present}
\begin{position}
\end{position}


%---------------------------------------------------------------------------

\section{\bf Employment}

\begin{format}
\employer{l}\location{r}\\
\title{l}\dates{r}\\
\body\\
\end{format}

\title{Consultant}
\employer{Insignia Environmental}
\location{Palo Alto, CA}
\dates{March 2009}
\begin{position}
    Bootstrap-based analysis of three experiments designed to estimate
    the environmental impact of installing new wind turbines in
    Contra Costa County, in California.
\end{position}

\title{Consultant}
\employer{Kingsford Capital Management}
\location{Pt. Richmond, CA}
\dates{August 2008--present}
\begin{position}
    General statistics advice, help with data analysis and programming in R.
\end{position}

\title{Biostatistician}
\employer{Stanford School of Medicine}
\location{Stanford, CA}
\dates{2006--2007}
\begin{position}
    Data analysis and brief write-ups for studies performed by resident
    physicians under the supervision of Dr. Raj Mitra, Orthopaedic Surgery 
    Department.
\end{position}

%------------------------------------------------------------------------------

\section{\bf Publications and Presentations}

``Minimax rank estimation for subspace tracking.''
With P.\ J.\ Wolfe.
\textit{Journal of Selected Topics in Signal Processing},
to appear.

``A rotation test to verify latent structure.''
With A.\ B.\ Owen.
\textit{Journal of Machine Learning Research},
\textbf{11}:603--624, 2010.

``Cross-validation for unsupervised learning.''
Supervised by A.\ B.\ Owen.
PhD thesis, Stanford University. 2009.

``Bi-cross-validation of the SVD and the non-negative matrix factorization.''
With A.\ B.\ Owen.
\textit{Annals of Applied Statistics},
\textbf{3}(2):564--594, 2009.

``Latent covariate detection and verification.''
With A.\ B.\ Owen, 
Biomedical Computation at Stanford (BCATS) Symposium, Stanford CA, October 21, 2006.
Poster.

``Latent covariate detection and verification: a case study in genetics.''
With A.\ B.\ Owen, SAMSI Random Matrices Opening Workshop, Chapel Hill, NC, 
September 17--20, 2006. Poster. 

``Application of simplicial sets to computational topology.''
Supervised by G.\ E.\ Carlsson.
Undergraduate honors thesis, Stanford University. 2003.

\section{\bf Working Papers}

``A graph log-linear model for characterizing repeated interactions.''
With P.\ J.\ Wolfe.


%------------------------------------------------------------------------------

%%===========================================================================%%
%\newpage
%\opening

\section{\bf Software}

{\texttt{RMTstat}}. With I.\ Johnstone, Z.\ Ma, and M.\ Shahram.
An \texttt{R} package for distributions and statistics from Random Matrix Theorey (RMT),
mostly related to large sample covariance matrices.
\href{http://cran.r-project.org/web/packages/RMTstat/index.html}http://cran.r-project.org/web/packages/RMTstat/index.html.

{\texttt{bcv}}.
An \texttt{R} package for bi-cross-validation.  This package implements
cross-validation-based methods for choosing the rank of an SVD approximation.
\href{http://cran.r-project.org/web/packages/bcv/index.html}http://cran.r-project.org/web/packages/bcv/index.html.

{\texttt{hs-blas}}.
A high-level interface to the BLAS linear algebra library for the Haskell
programming language.
\href{http://hackage.haskell.org/cgi-bin/hackage-scripts/package/blas}http://hackage.haskell.org/cgi-bin/hackage-scripts/package/blas.

{\texttt{hs-monte-carlo}}.
A Haskell library for performing Monte-Carlo simulations.
\href{http://hackage.haskell.org/cgi-bin/hackage-scripts/package/monte-carlo}http://hackage.haskell.org/cgi-bin/hackage-scripts/package/monte-carlo.

{\texttt{PLEX}}. With V.\ de Silva.
A collection of \textsc{MATLAB} routines developed as a research tool for building and 
studying simplicial complexes generated from point-cloud data.  Primarily written in C++.  
\href{http://comptop.stanford.edu/programs/plex.html}http://comptop.stanford.edu/programs/plex.html.


%------------------------------------------------------------------------------

\section{\bf Peer-Review Experience}

Annals of Applied Statistics

Annals of Statistics

Journal of the Bernoulli Society

Stochastic Processes and their Applications


\section{\bf Conferences and Workshops}

Sparse Random Structures: Analysis and Computation. Banff International
Research Station, Canada. January 24--29, 2010.

Joint Statistical Meetings 2007. Salt Lake City, UT. July 29 -- August 2, 2007.

The 3rd Workshop on Monte Carlo Methods. Cambridge, MA. May 13--14, 2007.

AIM Workshop on Random Matrices and Higher Dimensional Inference. Palo Alto, CA. April 10--13, 2007.

Biomedical Computation at Stanford (BCATS). Stanford CA. October 21, 2006.

SAMSI Random Matrices Opening Workshop. Chapel Hill, NC.  September 17--20, 2006.

DARPA Topological Data Analysis (TDA) Meeting. Santa Barbara, CA. May 8--10, 2006.

DARPA TDA Meeting. San Rafael, CA. October 29--30, 2005.

%\section{\bf Presentations}
%
%``Solving Nonlinear Wave Equations with Mathematica,''
%North Carolina State University, April 23, 2003.

%---------------------------------------------------------------------------

%\section{\bf Service}
%
%\begin{itemize}
%\item Vice President, Society of Undergraduate Mathematics, North
%  Carolina State University, 2001--2003
%\item Vice President, Economics Graduate Student Council, Duke
%  University, 2006--2007
%\end{itemize}

%%---------------------------------------------------------------------------%%

%\begin{center}
%\vspace{\fill}\ \newline
%{\tiny \rm $ $RCSfile: cv-us.tex,v $ $ }
%{\tiny \rm $ $Date: 2006/12/12 22:53:52 $ $ }
%{\tiny \rm $ $Revision: 1.28 $ $ }
%\end{center}

\end{resume}

\end{document}

%%===========================================================================%%
